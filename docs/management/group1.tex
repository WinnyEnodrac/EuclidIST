%\documentclass{Latex_Euclid_style}

%\usepackage{amsmath}
%\usepackage{amsfonts}
%\usepackage{amssymb}
%\usepackage{graphicx}
%\usepackage{epsfig}
%\usepackage[hang,small,bf]{caption}
%\usepackage{rotating}


% --------------

%\begin{document}

\section{Splinter Group 1: codes}

\subsubsection*{List of participants}
\begin{itemize}
 \item Valeria Pettorino
 \item Matteo Martinelli
\end{itemize}

\subsubsection*{Aim}
Make a recommendation on existing codes for Atomistic Forecasting approach (likelihood).\\
Expected output:
\begin{itemize}
 \item A technical note explaining pros and cons of the various potential public codes for this purpose 
 \item How do they meet our requirements
 \begin{itemize}
  \item On the division/requirements setting at fine-grained level 
  \item On individual probe work and on combinations
 \end{itemize}
 \item Make a consensus of codes, and make a short presentation showing initial pros and cons.
\end{itemize}

\subsection{Boltzmann codes}

\subsubsection{Overview of publicly available codes}



\paragraph{CAMB}
(\url{http://camb.info/})\\
{\it Language:} fortran90\\
{\it Includes:}\\
\begin{itemize}
 \item standard neutrinos, sterile neutrinos
 \item CDM
 \item DE: $w_0\ w_a$
\end{itemize}
{\it Pros: }
\begin{itemize}
%publicly available
 \item updated very often
 \item tested by a large community
 \item there are several implementations for MG models (though not as well tested)
 \item python wrapper for CAMB to join to any MCMC
 \item documentation available and often updated: quick feedback by Anony Lewis and large community, via cosmocoffee
\end{itemize}
{\it Cons:}
\begin{itemize}
 \item structure of the code requires to modify different parts of the code for different models
 \item no general distribution function for neutrinos
 \item only synchronous gauge
 \item maintainer (Antony Lewis) not currently in Euclid
\end{itemize}


\paragraph{EFTCAMB}(\url{http://wwwhome.lorentz.leidenuniv.nl/~hu/codes/})\\
{\it Language:} fortran90\\
{\it Includes:}\\
{\it Pros:}
\begin{itemize}
 \item based on CAMB
 \item very general parametrization for effective field theory (EFT) parametrisation of MG and DE models
 \item tested against HiCLASS/MGCAMB/private codes for some models
 \item several parameterisations of the background expansion and also of the perturbations (now including the Horndeski functions)
 \item perturbation evolution computed self-consistently for EFT class of models
 \item Very clearly written, structured, documented. 
 \item It has a MCMC wrapper.
\end{itemize}
{\it Cons: }
\begin{itemize}
 \item need further testing
 \item needs some understand stability conditions
 \item MG starts from z = 100 (not tested earlier)
\end{itemize}


\paragraph{MGCAMB}(\url{http://aliojjati.github.io/MGCAMB/})\\
{\it Language:} fortran90\\
{\it Includes: }
\begin{itemize}
 \item changes the response of the scalar metric potentials (phi and psi) to the perturbations: this can be due to modifications of the Einstein equations 
 (modified gravity) or due to extra clustering components (dark energy)
\end{itemize}
{\it Pros: }
\begin{itemize}
 \item it works with some well tested parametrization; easy to compare with previous results 
 \item can test  very general deviations from LCDM visible in the scalar metric perturbations
\end{itemize}
{\it Cons: }
\begin{itemize}
 \item When used as a ‘EFT replacement’:
 \begin{itemize}
  \item does not evolve all perturbation equations 
  \item assumes quasi static limit
 \end{itemize}
 \item General deviations in scalar metric perturbations require careful parameterisation, some further developments (eg. Gaussian process) may be required
 \item Does not include similar deviations in vector and tensor perturbations
 \item Works with $\Lambda$CDM background
\end{itemize}
Warning: the code was recently updated (late february). These comments do not refer to the latest version.\\


\paragraph{CLASS}(\url{http://class-code.net/})\\

{\it Language:} C

{\it Includes:}
\begin{itemize}
 \item various forms of neutrinos and dark matter
 \item DE with a given equation of state, quintessence with scalar potentials.
 \item DE extension is HiClass but not public (Bellini, Miguel Zuma)
 \item newtonian and synchronous gauge
\end{itemize}
{\it Pros: }
\begin{itemize}
 \item structure of code
 \item python wrapper
 \item two gauges (newtonian and synchronous: except the initial conditions which are only written in the synchronous gauge and gauge-transformed). 
 \item maintainer (Julien Lesgourgues) is active Euclid member
\end{itemize}
{\it Cons: }
\begin{itemize}
 \item community that uses/tested CLASS is smaller than CAMB
\end{itemize}





\subsubsection{Requirements to trust a code in MG}

For every MG model, there should be at least 2 Boltzmann codes, agreeing. Agreement is defined as follows: 
\begin{itemize}
 \item check them on LCDM, need to agree at less than 0.1\%
 \item check them in a MG scenario in different ranges of parameters: they should match at the same level at which they agree in LCDM on 
all background quantities, all perturbation quantities, all spectra
\end{itemize}

 




\subsection{Parameter estimation codes}

\subsubsection{Requirements for parameter estimation codes}

The basic requirement for codes which obtain cosmological predictions from Boltzmann codes and perform parameter estimation is ``modularity''.\\
A modular code is easy to modify in case some changes and updates need to be included, e.g. additional likelihoods, extra parameters, correction to observables (systematics).\\
Furthermore, separate modules for different physical quantities are needed for the atomistic approach as they allow to easily retrieve and compute the intermediate quantities of the theory
flows (see e.g. Fig. \ref{WLtomo}) and to compare them between different codes.\\
However too much modularity may affect code performances.\\

A wishlist of module is:
\begin{enumerate}
 \item module calling the Boltzmann/transfer function codes for primary science
 \item module calling the sampler
 \item module for each likelihood for each probe
 \item module for background quantities
 \item module for perturbation quantities
 \item module for non-linear corrections (ex. halofit)
 \item module that gets the standard output from Boltzmann codes and produces list of suitable output
 \item postprocessing of the samples
 \item plotting modules 
\end{enumerate}

In the following, the publicly available codes are listed, together with the requirements which are satisfied by each of them. The missed requirements are also listed and the complexity 
of implementation is indicated as easy (E), medium (M) and difficult (D)

\subsubsection{Overview of publicly available codes}

\paragraph{COSMOMC}(\url{http://cosmologist.info/cosmomc/})\\
{\it Language:} fortran90\\
{\it pros:}
\begin{itemize}
 \item updated constantly
 \item very well tested, very efficient for standard cosmology
 \item used by a large community
 \item easy to add parameters
 \item easy to add an additional likelihood by adding a separate module
 \item nice python graphical interactive interface for first tests
 \item well documented
 \item there are many options and more control to analyse the samples. 
 \item Updated to all currently available data (WL, GC, clusters, CMB).
 \item Likelihood Sampler works really well
\end{itemize}
{\it Cons: }
\begin{itemize}
 \item if you increase the number of parameters, the sampler gets slower. Other samplers are publicly available and already interfaced with COSMOMC. However they are not included in the 
 standard version (interface needs to be modified when COSMOMC is updated)
 \item maintainer (Antony Lewis) not currently in Euclid
\end{itemize}
{\it Wish-list requirements: }
\begin{itemize}
 \item included: 1, 2 (add samplers), 3, 6, 8, 9
 \item missing: 4 (E CAMB has background functions collected in module ModelParams in modules.f90 which can be called by COSMOMC), 5 (D), 6 (alternatives to halofit), 7 (E/M) 
\end{itemize}

\paragraph{MONTEPYTHON}(\url{https://github.com/baudren/montepython_public} or \url{http://baudren.github.io/montepython.html})\\
{\it Language:} python\\
{\it Includes:}
\begin{itemize}
 \item Euclid mock likelihood (weak lensing + galaxies) is publicly available
 \item CMB, SNIa, BAO, WiggleZ (CHFTLens and SDSS to be released); 
\end{itemize}
{\it Pros: }
\begin{itemize}
 \item modular environment
 \item can switch among different samplers
 \item it has a graphical interface
 \item can easily be interfaced with different Boltzmann codes
 \item can use MPI
 \item extra parameters need to be included only in the input file
 \item easy to add a new likelihood.
\end{itemize}
{\it Cons:} 
\begin{itemize}
 \item community that uses/tested MP is smaller than COSMOMC
 \item less frequently updated 
\end{itemize}
{\it Wish-list requirements: }
\begin{itemize}
 \item included: 1 (for generic Boltzmann code; currently points at Class wrapper; Camb wrapper in preparation), 2 (MH, Multinest, CosmoHammer, importance sampling, reprocessing chains, fisher matrices), 3, 4, 5, 6 (Halofit), 7, 8, 9 (can choose between MontePython or Getdist to produce plots or reprocess: chains are in the same format as cosmomc and can use 8,9 from COSMOMC) 
 \item missing: 6 (alternatives to Halofit E/M) 
\end{itemize}

\paragraph{COSMOSIS}(\url{https://bitbucket.org/joezuntz/cosmosis/wiki/Home})\\
{\it Language:} ??\\
{\it Includes: }
\begin{itemize}
 \item sampler code (http://arxiv.org/pdf/1409.3409v1.pdf)
 \item CFHTLens, Planck, some BAO, SNae JLA. 
\end{itemize}
{\it pros:} 
\begin{itemize}
 \item can be wrapped to CAMB and CLASS
 \item it is modular (can introduce other samplers or Boltzmann codes) 
 \item can add modules in different languages
 \item new parameters are defined only in the input parameter file
 \item They will include SDSS. Easy to implement a new module for a new likelihood.
 \item Used by DES.
\end{itemize}
{\it cons:} 
\begin{itemize}
 \item not very well tested
 \item small community uses it
 \item who is updating it?
\end{itemize}
{\it Wish-list requirements: }
\begin{itemize}
 \item included: 1 (generic Boltzmann code: CAMB, CLASS), 2 (MH, MultiNest, EMCEE), 3 (...add…), 4 (?), 6 (Halofit, Halofit + T correction), 7
 \item Datablock the object passed down the pipeline. For a given set of parameters all module inputs are read from the datablock and all module outputs are written to it 
 \item missing: 4 (E but not implemented?), 5 (?), 6 (alternatives to Halofit; E/M), 8 (format of chains? can one use getdist? how advanced as compared to getdist?), 9 (how advanced with respect to COSMOMC ones?) 
\end{itemize}
\paragraph{NumCosmo}(\url{http://www.nongnu.org/numcosmo/})\\




\paragraph{emcee}(\url{http://dan.iel.fm/emcee/current/})\\
{\it Language:} python\\
{\it Includes:}
\begin{itemize}
 \item 3 different samplers (unified calling interface) and plotting tools
\end{itemize}
{\it Pros:} 
\begin{itemize}
 \item affine invariant sampling for fast sampling of degenerate likelihoods
 \item parallel tempered sampler for multimodal likelihoods
 \item standard Metropolis-Hastings sampler
 \item many plotting tools included
 \item can use MPI
 \item interfaces reasonbly easy to write
 \item interface to lensing code exists (in C)
 \item interface to supernova code exists (in C or python)
 \end{itemize}
{\it Cons:} 
\begin{itemize}
 \item not specifically interfaced to a Boltzmann code
\end{itemize}
{\it Wish-list requirements: }
\begin{itemize}
 \item inclusion of observables CMB, CMB-lensing, CMB cross-correlations (iSW) and galaxy clustering
 \item call to Boltzmann code (Bjoern: we’re working to add CMB-related probes)
\end{itemize}


\subsection{Code recommendation}

\paragraph{Coding Language}
The SGS use python and C++ only, and it would be good for us to follow them to have some coherence. Although there are pros and cons.\\
It would be ideal to have a python infrastructure: wrappers in python to other codes written in different open sources free languages (instead of IDL, Mathematica)\\
FR software to convert languages from one to another: help to write a python wrapper, for example SDC is using swig (http://www.swig.org/).
Good experience with SWIG, it serves as a way of passing data between python and other languages (Bjoern: tried out python$\rightarrow$C with openMP-parallelisation)\\
Make sure all libraries used/needed are available. 
SDC already has a list of libraries that are allowed to be used, at \url{http://euclid.roe.ac.uk/projects/codeen-users/wiki/EDEN}.
Python slow for high level performances, 

%\paragraph{Infrastructure}

\paragraph{Boltzmann code}

No specific choice, but recommendation is that we need at least 2 codes that match for each model (match in the sense explained above); 
in addition, our recommendation is to use MGCAMB only for tests (and clearly specify limitations of the code and assumptions in it)


\paragraph{Parameter estimation}
MontePython at present is the one that includes the highest number of requirements in the wish list. JL \& team within Euclid guarantee fast 
update for wrappers to Boltzmann code version updates and to new data which become available. 

\paragraph{Analysis and visualization}
Getdist and plotting tools provided by cosmomc

\paragraph{Code testing}
Self-consistent check of the likelihood codes:  use a simulation to produce a set of data and check if we can recover the initial cosmological model.\\
Available simulations: \\
galaxy mock catalogue
\begin{enumerate}
 \item from MICE simulations (HOD) in LCDM cosmology  \url{http://internal.euclid-ec.org/?page_id=1551}
 \item from Durham simulations (SAM) in LCDM cosmology 
\end{enumerate}
Needed:
\begin{itemize}
 \item simulations from non-standard cosmological models: e.g. f(R), DE 
 \item mock catalogue for WL (to be asked to the WL splinter-2
\end{itemize}


\paragraph{Required output from other groups}
\begin{itemize}
 \item SDC: list of python libraries (\url{http://euclid.roe.ac.uk/projects/codeen-users/wiki/EDEN})
 \item TWG:
 \begin{itemize}
  \item parameter definition document
  \item list of Boltzmann codes available for MG
 \end{itemize}
 \item Simulations: list of simulations available
\end{itemize}





%\end{document}



